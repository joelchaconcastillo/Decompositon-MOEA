Premature convergence is one of the most typical drawbacks of \EAS{}.
%
\MOEAS{} indirectly promote the preservation of diversity in the variable space
because of the implicit relationship between the diversity maintained in the objective space and the
one maintained in the variable space.
%
However, for many problems the degree of diversity maintained is not sufficient to ensure the exploratory
power of genetic operators and locate the optimal regions.
%
In single-objective optimization, many of the state-of-the-art algorithms explicitly manage
diversity.
%
Specifically, those schemes that relate the degree of diversity to the elapsed period of execution 
and to the stopping criterion
have excelled.
%
This paper shows that this design principle is also helpful in the area of multi-objective optimization,
where the optimization of many of the most complex popular benchmark problems can be improved further
by applying this design principle.

In order to prove this hypothesis, a novel replacement operator based on the aforementioned design principle
is applied to generate a decomposition-based \MOEA{} that takes into account the diversity in both the variable 
and objective spaces.
%
This is done using a dynamic penalty method. 
%
Note that since the aim of the approach is to improve the results when considering metrics in the objective space,
the importance given to the diversity in the variable space is reduced as the evolution progresses, meaning that
in the later phases, our proposal behaves more similarly to traditional \MOEAS{}.
%
Additionally, taking into account recent advances, and to ensure that our proposal maintains
high-quality solutions despite the penalty scheme, an external archive based on the
R2-indicator is incorporated.
%
Because of this, we refer to our proposal as \textit{Archived Variable Space Diversity MOEA based on Decomposition} (\AVSDMOEAD{}).

The experimental validation carried out shows the remarkable improvement provided by \AVSDMOEAD{} in comparison to
state-of-the-art \MOEAS{} with both two-objective and three-objective problems.
%
The scalability analyses show that as the number of decision variables increases, the benefits of including
proper diversity management are even more important, so the differences in performance increase.
%
In fact, the most remarkable benefits emerge for the most complex cases.
%
Moreover, the analysis of the initial penalty threshold, which is an additional parameter required by \AVSDMOEAD{}, 
shows that the method is quite robust, which makes finding a proper parameter value an easy task.
%
Finally, in order to better understand the reasons behind the huge superiority of our proposal, some analyses involving
the dynamics of the populations are provided.
%
In comparison to state-of-the-art algorithms, our proposal clearly slows down convergence.

In the future, we plan to apply the principles studied in this paper to other categories of \MOEAS{}.
%
For instance, including the diversity management method presented in this paper in indicator-based \MOEAS{} seems plausible.
%
Additionally, in order to obtain even better results, these strategies are going to be incorporated with continuation and/or individual
improvement methods.
