\subsection{Diversity in Evolutionary Algorithms}

The proper balance between exploration and exploitation is one of the keys to designing a successful \EAS{}.
%
In the single-objective domain, it is known that properly managing the diversity of the variable space is a way to achieve this balance,
and as a consequence, a large number of diversity management techniques have been devised~\cite{pandey2014comparative}.
%
Specifically, these methods are classified depending on the component(s) of the \EA{} that is modified to alter how much diversity is maintained.
%
A popular taxonomy identifies the following groups~\cite{Joel:Crepinsek}: \textit{selection-based}, \textit{population-based}, 
\textit{crossover/mutation-based}, \textit{fitness-based}, and \textit{replacement-based}.
%
Additionally, the methods are referred to as \textit{uniprocess-driven} when a single component is altered, whereas the term
\textit{multiprocess-driven} is used to refer to those methods that act on more than one component.

Among the previous proposals, the replacement-based methods have yielded very high-quality results in recent years~\cite{segura2016improving}, so
this alternative was selected with the aim of designing a novel \MOEA{} that explicitly incorporates a way to control the diversity 
of the variable space.
%
The basic principle of these methods is to bias the level of exploration in successive generations by 
controlling the diversity of the survivors~\cite{segura2016improving}.
%
Since premature convergence is one of the most common drawbacks in the application of \EAS{}, 
modifications are usually performed with the aim of slowing down the convergence.
%
One of the most popular proposals belonging to this group is the \textit{crowding} method, which
is based on the principle that offspring should replace similar individuals from the previous generation~\cite{mengshoel2014adaptive}.
%
Several replacement strategies that do not rely on crowding have also been devised.
%
In some methods, diversity is considered as an objective.
%
For instance, in the hybrid genetic search with adaptive diversity control (\HGSADC{})~\cite{vidal2013hybrid}, individuals are sorted 
by their contribution to diversity and by their original cost.
%
Then, the rankings of the individuals are used in the fitness assignment phase.
%
A more recent proposal~\cite{segura2016improving} incorporates a penalty approach to gradually alter the amount of diversity 
maintained in the population.
%
Specifically, the initial phases preserve a higher amount of diversity than the final phases of the optimization.
%
This last method has inspired the design of the novel proposal put forth in this paper for multi-objective optimization.
%

It is important to remark that in the case of multi-objective optimization, little work related to maintaining the 
diversity of the variable space has been done.
%
The following section reviews some of the most important \MOEAS{} and introduces some of the works that consider
the maintenance of diversity of the variable space.


\subsection{Related Works}

In the last decade few MOEAs were specifically designed to address the diversity in the decision variables space.
%
Although that the diversity in single-objectives EAs is a matter of importance refs, in multi-objective optimization usually the diversity in the variable space is ignored.
%
This might occurs, since the objectives are usually in conflict, therefore often is maintained a diversity level in the decision space.
%
Also, the decision space is disregarded, since that at the end of the optimization process the quality of the solutions relies only in the objective space.
%
In single-objective optimization, high quality solutions have been provided since that a balance between exploration and exploitation is reached through the optimization process refs.
%
In this way, the premature convergence, which is considered as a drawback, can be avoided.
%
Some strategies in single-objective optimization to avoid this drawback is explicitly induce a the diversity considering the criteria stop, thus at first stages the exploration levels are promoted and at the end the exploitation of the promising regions is induced.
%
A similar issue is addressed in multi-objective problems, in such a way that the evolutionary search is stagnated, and only are explored the same region.
%
Particularly, the idea to integrate decision space diversity into the optimization has been proposed in 1994 with the first NSGA work REF.
%
In this last work the decision vectors are considered into the fitness sharing procedure.
%
Thereafter, the most algorithms concentrated in the objectives space only.
%
Alternatively, several approaches with MOPs related directly in decision space has arisen.
%
These approaches, further as the usual MOPs, also aims to provide diverse solutions in the decision space.
%
Principally, based in that there exists a variety of problems where the image of the Pareto Front corresponds to several distributions in the Pareto Set.
%

In 2003 GDEA \cite{toffolo2003genetic} integrated diversity into the search as an additional objective.
%
This MOEA introduced by Toffolo and Benini invoked two selection criteria, non-dominated sorting as the primary one and a metric for decision space diversity as the secondary one.
%
In 2005, Chan and Ray \cite{chan2005evolutionary} suggested to use two selection operators in MOEAs; one encourages the diversity in the objective space and the other does so in the decision space.
%
They implemented KP1 and KP2, two algorithms using these two selection operators.
%
After that, in 2008, the Omni-optimizer \cite{deb2008omni} was developed, which extends the original idea of the NSGA. 
%
Particularly, the diversity measure take both the decision and the objective space diversity into account, Omni-optimizer first uses a rank procedure, were the objective space measure is always considered first, and only if there are ties the diversity in decision space is taken into consideration.
%
However, the drawback of this approach is that the diversity plays an inferior role and there is no possibility to change the tradeoff between the diversity measures.
%
In 2009, were proposed the  CMA-ES niching framework \cite{shir2009enhancing} , and the probabilistic Model-based Multi-objective Evolutionary Algorithm (MMEA)\cite{zhou2009approximating} .
%
The first, extend a niching framework to include the diversity in the space diversity.
%
The second, applies a clustering procedure in objective space and then builds a model from the solutions in these clusters.
%
In 2010, was proposed the Diversity Integrating Hypervolume-based Search Algorithm (DIVA) \cite{ulrich2010integrating}, this algorithm introduces a method to integrate decision space diversity into the hypervolume indicator, such that these two set measures can be optimized simultaneously.

\subsection{Decomposition-Based MOEAs}


In the last decades \MOEAS{} have gained enough popularity dealing with \MOPS{}, given their outstanding performance a vast amount of variants have been designed.
%
To better classify the different schemes, several taxonomies have been proposed~\cite{bechikh2016recent}.
%
A well known classification can be based on Pareto dominance, indicators and/or decomposition~\cite{trivedi2016survey}.
%
The domination-based \MOEAS{} are based on the application of the Pareto dominance relation and techniques to promote the diversity in the objective space~\cite{deb2002fast}.
%
Similarly, the indicator-based \MOEAS{} incorporate a measure of quality of the approximations attained by the \MOEAS{}~\cite{beume2007sms}.
%
A recent \MOEA{} that belongs to this category is the R2-Indicator-Based Evolutionary Multi-objective Algorithm (\RMOEA{})~\cite{trautmann2013r2}, whose performance in \MOPS{} has been quite promising.
%

%
Finally, the decomposition based algorithms which implements scalarizing functions to transform the \MOP{} into several single-objective optimization subproblems and are simultaneously solved in a single run.
%
The transformation can be in several ways, some of the most known are the weighted sum approach, the weighted tchebycheff approach and the penalty-based boundary intersection approach.
%
Therefore, a set of weight vectors defines different single-objective functions, which are optimized by the \MOEA{}.
%
In addition, the weight vectors are selected a priori with the aim to obtain well-spread solutions among the Pareto front, however the optimal selection of weights depends of each problem and its Pareto-Geometry.
%


Although that one of the most popular decomposition-based algorithms is the \MOEAD{} proposed by Zhang et. al~\cite{zhang2007moea} there are several antecedents of metaheuristics that implements the idea of decomposition for solving \MOPS{}~\cite{ishibuchi1998multi} and~\cite{murata2002cellular}.
%
Particularly, \MOEAD{} has its origins in the cellular multi-objective genetic algorithm (C-MOEA) proposed by Murata and Gen~\cite{murata2002cellular}.
%
One special feature of the \MOEAD{} is the definition of neighborhoods.
%
Each subproblem is associated with the $k$-nearest subproblems in terms of the distance to the weight vectors conforming a neighborhood.
%
Another feature of the \MOEAD{} is the mating selection and replacement mechanism incorporated in each neighborhood.
%
Those features have implications in the diversity preserved in the population by the algorithm.
%
Specifically, a single good solution can replace several inferior neighboring solutions that can result in deterioration of the population diversity~\cite{wang2015constrained}.
%
To alliviate the previous shortcomming and stablishing several improvements Li and Zhang proposed \MOEADDE{}~\cite{li2009multiobjective}.
%
The main changes imposed by the \MOEADDE{} are the incorporation of \DE{} operators, computational resource allocation strategy, mating selection and replacement mechanism.
%
In this decomposition algorithm, the simulated binary crossover (\SBX{}) that is employed in the general \MOEAD{} is replaced by the \DE{} operators.
%
However to promote variation in the population the polynomial mutation~\cite{deb2001multi} is still used.
%
This adjustment is given that the \DE{} operator can be invariant of any orthogonal coordinate rotation being ideal for dealing with complicated \PS{}.
%
In addition, in the \MOEAD{} all the subproblems are treated equally reciving the same computational effort by generation.
%
However, depending in the complexity and the Pareto shape of the functions, each subproblem might require a different computational effort, therefore \MOEADDE{} incorporates a computational resource allocation strategy.
%
In order, given that the replacement mechanism of the \MOEAD{} is too elitist, the \MOEADDE{} incorporates an extra measure for maintaining the population diversity in the mating selection, thus given a probability this mechanism selects three parent solutions from the neighborhood or the whole population, enhancing the exploration ability.
%
Finally, in the \MOEAD{} the maximal number of solutions replaced by a solution could be as large as the neighborhood size reducing dramatically the diversity among the population.
%
To overcome this shortcoming the \MOEADDE{} incorporates a parameter to limit the maximal number of solutions replaced by a child solution, avoiding to many copies in the population.
%

