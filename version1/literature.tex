\subsection{Decomposition-Based MOEAs}

Several \MOEAS{} have been proposed in the literature and can be classified in the following three categories~\cite{trivedi2016survey}:
%
\textit{Dominance-Based Framework}, \textit{Indicator-Based Framework}, and \textit{Decomposition-Based Framework}.
%

Particularly, decomposition based algorithms implement scalarizing functions to transform the \MOP{} into several sigle-objective optimization subproblems that are simultaneously solved in a single run.
%
The transformation can be in several ways, some of the most known are \textit{the Weighted Sum Approach}, \textit{the weighted Tchebycheff Approach} and \textit{the Penalty-Based Boundary Intersection Approach}.
%


One of the most popular decomposition-based algorithms is the one proposed by Zhang et. al~\cite{zhang2007moea}, however there are several antecedents of metaheuristics that implements the idea of decomposition for solving \MOPS{} REF10 and REF11.
%
In addition, several works stablished that the original \MOEAD{} has it origins in the cellular multi-objective genetic algorithm (C-MOEA) proposed by Murata and Gen REF11.
%



Previously that one of the most popular version of decomposition
\subsubsection{Original MOEA/D Framework}



\subsection{Diversity in Evolutionary Algorithms}

The proper balance between exploration and exploitation is one of the keys to designing a successful \EAS{}.
%
In the single-objective domain, it is known that properly managing the diversity of the variable space is a way to achieve this balance,
and as a consequence, a large number of diversity management techniques have been devised~\cite{Mohan:14}.
%
Specifically, these methods are classified depending on the component(s) of the \EA{} that is modified to alter how much diversity is maintained.
%
A popular taxonomy identifies the following groups~\cite{Joel:Crepinsek}: \textit{selection-based}, \textit{population-based}, 
\textit{crossover/mutation-based}, \textit{fitness-based}, and \textit{replacement-based}.
%
Additionally, the methods are referred to as \textit{uniprocess-driven} when a single component is altered, whereas the term
\textit{multiprocess-driven} is used to refer to those methods that act on more than one component.

Among the previous proposals, the replacement-based methods have yielded very high-quality results in recent years~\cite{Segura:17}, so
this alternative was selected with the aim of designing a novel \MOEA{} that explicitly incorporates a way to control the diversity 
of the variable space.
%
The basic principle of these methods is to bias the level of exploration in successive generations by 
controlling the diversity of the survivors~\cite{Segura:17}.
%
Since premature convergence is one of the most common drawbacks in the application of \EAS{}, 
modifications are usually performed with the aim of slowing down the convergence.
%
One of the most popular proposals belonging to this group is the \textit{crowding} method, which
is based on the principle that offspring should replace similar individuals from the previous generation~\cite{Mengshoel:14}.
%
Several replacement strategies that do not rely on crowding have also been devised.
%
In some methods, diversity is considered as an objective.
%
For instance, in the hybrid genetic search with adaptive diversity control (\HGSADC{})~\cite{Vidal:13}, individuals are sorted 
by their contribution to diversity and by their original cost.
%
Then, the rankings of the individuals are used in the fitness assignment phase.
%
A more recent proposal~\cite{Segura:17} incorporates a penalty approach to gradually alter the amount of diversity 
maintained in the population.
%
Specifically, the initial phases preserve a higher amount of diversity than the final phases of the optimization.
%
This last method has inspired the design of the novel proposal put forth in this paper for multi-objective optimization.
%

It is important to remark that in the case of multi-objective optimization, little work related to maintaining the 
diversity of the variable space has been done.
%
The following section reviews some of the most important \MOEAS{} and introduces some of the works that consider
the maintenance of diversity of the variable space.


\subsection{Related Works}

In the last decade few MOEAs were specifically designed to address the diversity in the decision variables space.
%
Although that the diversity in single-objectives EAs is a matter of importance refs, in multi-objective optimization usually the diversity in the variable space is ignored.
%
This might occurs, since the objectives are usually in conflict, therefore often is maintained a diversity level in the decision space.
%
Also, the decision space is disregarded, since that at the end of the optimization process the quality of the solutions relies only in the objective space.
%
In single-objective optimization, high quality solutions have been provided since that a balance between exploration and exploitation is reached through the optimization process refs.
%
In this way, the premature convergence, which is considered as a drawback, can be avoided.
%
Some strategies in single-objective optimization to avoid this drawback is explicitly induce a the diversity considering the criteria stop, thus at first stages the exploration levels are promoted and at the end the exploitation of the promising regions is induced.
%
A similar issue is addressed in multi-objective problems, in such a way that the evolutionary search is stagnated, and only are explored the same region.
%
Particularly, the idea to integrate decision space diversity into the optimization has been proposed in 1994 with the first NSGA work REF.
%
In this last work the decision vectors are considered into the fitness sharing procedure.
%
Thereafter, the most algorithms concentrated in the objectives space only.
%
Alternatively, several approaches with MOPs related directly in decision space has arisen.
%
These approaches, further as the usual MOPs, also aims to provide diverse solutions in the decision space.
%
Principally, based in that there exists a variety of problems where the image of the Pareto Front corresponds to several distributions in the Pareto Set.
%

In 2003 GDEA \cite{toffolo2003genetic} integrated diversity into the search as an additional objective.
%
This MOEA introduced by Toffolo and Benini invoked two selection criteria, non-dominated sorting as the primary one and a metric for decision space diversity as the secondary one.
%
In 2005, Chan and Ray \cite{chan2005evolutionary} suggested to use two selection operators in MOEAs; one encourages the diversity in the objective space and the other does so in the decision space.
%
They implemented KP1 and KP2, two algorithms using these two selection operators.
%
After that, in 2008, the Omni-optimizer \cite{deb2008omni} was developed, which extends the original idea of the NSGA. 
%
Particularly, the diversity measure take both the decision and the objective space diversity into account, Omni-optimizer first uses a rank procedure, were the objective space measure is always considered first, and only if there are ties the diversity in decision space is taken into consideration.
%
However, the drawback of this approach is that the diversity plays an inferior role and there is no possibility to change the tradeoff between the diversity measures.
%
In 2009, were proposed the  CMA-ES niching framework \cite{shir2009enhancing} , and the probabilistic Model-based Multi-objective Evolutionary Algorithm (MMEA)\cite{zhou2009approximating} .
%
The first, extend a niching framework to include the diversity in the space diversity.
%
The second, applies a clustering procedure in objective space and then builds a model from the solutions in these clusters.
%
In 2010, was proposed the Diversity Integrating Hypervolume-based Search Algorithm (DIVA) \cite{ulrich2010integrating}, this algorithm introduces a method to integrate decision space diversity into the hypervolume indicator, such that these two set measures can be optimized simultaneously.
