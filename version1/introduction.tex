\IEEEPARstart{M}{ulti-objective} Evolutionary Algorithms (\MOEAS{}) are one of the most popular approaches to deal with Multi-objective Optimization Problems (\MOPS{}).
%
MOEAs are usually employed in problems whose formulation is complicated or inaccessible.
%
A continuous box-constrained minimization MOP involves two or more conflicting objectives and is defined in Eq. (\ref{eqn:main})
%
\begin{equation}\label{eqn:main}
\begin{split}
&min \quad F(x) = (f_1(x), ..., f_M(x)) \\
&s.t. \quad x \in \Omega.
\end{split}
\end{equation}

where $\Omega \subseteq \Re^D$ denotes the decision space, $F: \Omega \rightarrow Y \subseteq \Re^M$ consists of $M$ objectives and $Y$ is the objective space.
%
Given two solutions $x_1, x_2 \in \Re^D$ is said that $x_1$ dominates $x_2$ denoted as $x_1 \prec x_2$ if and only if $f_i(x_1) \leq f_i(x_2)$ for all $i \in \{1,...,M\}$ and $f_i(x_1) < f_i(x_2)$ for at least one objective.
%
Therefore, $f_i(x_1)$ should be better or equal to $f_i(x_2)$ and $f_i(x_1)$ should be better for at least one objective.
%
A solution $F(x^*)$ is called a Pareto-optimal solution if there does not exist $F(x) \in Y$ such that $x \prec x^*$.
%
The set of all $x^* \in Y$ is called the Pareto-optimal solution set (\PS{}), and their image is the Pareto Front (\PF{}).
%
The goal of the \MOEAS{} is to find a set of solutions that are well-distributed and converged to the PF in the objective space~\cite{trivedi2016survey}.
%
%%%%%%%%%%%%%%%%%%%%%%%%%%%%%%%%%%%%%%%%%%%%%%%%%%%%% end paragraph

The Evolutionary Algorithms (\EAS{}) are one of the most popular meta-heuristics employed to deal with \MOPS{} due its capability to approximate several solutions in a single run~\cite{das2011real, zhou2011multiobjective}.
%
Through the last decades, the development of \MOEAS{} has grown dramatically~\cite{van1998multiobjective, coello2007mop}, resulting in algorithms that can be efficient with problems of different characteristics.
%
Some of them can be multimodality, deceptiveness and separability~\cite{huband2006review}.
%
In spite of the huge quantity of designed \MOEAS{}, unlike single-objective, there are several paths that have not been fully explored, perhaps the main reason of this are the effects of dealing with multiple objectives provoking the unability of the direct incorporation of several single-objective methods.
%
For instance, the mating restrictions and sharing strategies are similarly employed in single-objective than in multi-objective algorithms~\cite{fonseca1995multiobjective}, those strategies has been incorporated to \MOEAD{}~\cite{zhang2007moea} and \NSGAII{}~\cite{deb2002fast} respectively, two of the most popular \MOEAS{}~\cite{trivedi2016survey, zhou2011multiobjective}.
%
Nevertheless, the most representative \MOEAS{} still have similar drawbacks than in single-objective such as stagnation and premature convergence~\cite{ishibuchi2006comparison, castillo2017multi, buche2003self, lu2002dynamic}.
%

%%%%%%%%%%%%%%%%%%%%%%%%%%%%%%%%%%%%%%%%%%%%%%%%%%%%% end paragraph
One of the most critical aspects of the \EAS{} is to achieve an efficient search through a single-run, in which it should explore the landscape as best as possible.
%
The latter could be achieved through an ideal balance between exploration and intensification~\cite{lin2009auto}.
%
This balance is not easily carried out since it depends of the \MOP{}.
%
In fact, several authors have pointed out that the diversity based techniques should be taken into account as the corner stone to attain such balance~\cite{vcrepinvsek2013exploration}.
%
In this way, exploration and intensification could be diagnosed and managed through the diversity presented in the population.
%


%%%%%%%%%%%%%%%%%%%%%%%%%%%%%%%%%%%%%%%%%%%%%%%%%%%%% end paragraph
The mechanisms designed to deal with diversity seems to be the key to attain quality solutions in single-\EAS{}.
%
Perhaps, one of the most important benefits of promoting diversity is that it provides a way to deal with premature convergence and stagnation.
%
In fact, the diversity could be handled through several mechanisms such as the variation operator~\cite{herrera2003fuzzy},~\cite{mitchell1998introduction}, the replacement operator~\cite{segura2015novel} and/or population models~\cite{koumousis2006saw}.
%
Particularly, a version of the replacement operator which explicitly preserves diversity through several stages is one of the most promising~\cite{vcrepinvsek2013exploration}.
%
%%%%%%%%%%%%%%%%%%%%%%%%%%%%%%%%%%%%%%%%%%%%%%%%%%%%% end paragraph

Recently, some algorithms that yields to single-objective optimization have altered their behaviour-search depending of the stopping criterion and the elapsed function evaluations~\cite{castillo2019differential}.
%
Therefore, the combination of the stopping criterion and diversity techniques in long-term executions have been quite successfull in the most complex problems, in fact such principles have allowed to discover better optimal solutions~\cite{segura2016improving}.
%
These strategies explicitly handle the diversity of the population in several stages which takes place in realtion with the function evaluations elapsed and the stopping criterion~\cite{segura2015novel}.
%
For instance, in combinatorial domains new best-known solutions for some well-known variants of the frequency assignment problem~\cite{segura2016improving}, and for a two-dimensional packing problem~\cite{de2010optimisation}.
%
Similarly, this principle guided the design of the winning strategy at the Second Wind Fram Layout Optimization Competition, that was held in the Genetic and Evolutionary Conference~\cite{wilson2018evolutionary}.
%
Recently, in the case of continuous domains this replacement phase has been incorporated to Differential Evolution (\DE{})~\cite{castillo2019differential}, which attained remarkably superior results than the winners of the competition carried out in \IEEE{} Congress of Evolutionary Computation (\CEC{}) of the years 2016 and 2017.
%
Therefore, this novel principle of incorporating the replacement operator with the aim of explicitly control the diversity has given quite good results in both discrete (\NP{}-hard problems) and continuous domains.
%
However, the latter strategy requires long-term executions to attain the best benefits, in fact its application tends to be even more feasible given the constant growing of the computational power, which is not considerated in the most popular algorithms that might suffer some premature convergence or stagnation in long-term executions.
%
%%%%%%%%%%%%%%%%%%%%%%%%%%%%%%%%%%%%%%%%%%%%%%%%%%%%% end paragraph

The usual design of \MOEAS{} is especifically build to attain well-spread solutions and to cover the \PF{} (coverage), therefore some of them incorporate different mechanisms to achieve such goal.
%
However, most of the \MOEAS{} are quite immersed in the objective space and therefore the decision variable space is usually disregarded.
%
Nevertheless the latter space is crucial and should be taken into consideration along the evolutionary process.
%
In fact, preserving several levels of diversity in the variable space along the process might improve the final attained solutions.
%
Perhaps, one of the main challenges of promoting diversity in the \MOPS{} is that the diversity between spaces does not have an clear relation, i.e. inducing diversity in the variable space does not guarantee diversity and/or convergence in the objective space.
%
In addition, attaining well-spread solutions in the objective space might implicitly inherit some degree of diversity in the variable space.
%
Nevetheless, this diversity might not be enough and the reproduction operators could lose its exploratiory strength at some point of the process~\cite{lu2002dynamic}.
%
%%%%%%%%%%%%%%%%%%%%%%%%%%%%%%%%%%%%%%%%%%%%%%%%%%%%% end paragraph

In spite of the amazing amount of \MOEAS{} that have been developed, this paper proposes a novel and different \MOEA{} \textit{the Variable Space Diversity MOEA based in Decomposition} (\VSDMOEAD{}), which explicitly induces diversity in the variable space through several stages to obtain a proper balance from exploration towards intensification.
%
Particularly, the \VSDMOEAD{} is mainly based in a simplification of the \MOEADDE{}~\cite{zhang2009performance} and \CMOGA{}~\cite{murata2002cellular}.
%
While the former attained the first place in the CEC-09, the latter is one of the pioneer \MOEAS{} based in decomposition.
%
Specially, the \VSDMOEAD{} incorporates a replacement operator that takes into consideration the stopping criterion and the number of function evaluations.
%
In this way, this algorithm grants more importance to the diversity of the variable space at the initial stages and as the function evaluations evolve, it gradually grants less importance to the diversity of the variable space until a half of the total number of function evaluations.
%
Therefore, after the half of the total number of function evaluations the algorithm has a similar behavior than the state-of-the-art \MOEAS{}.
%
Since that in the literature exists a broad kind of \MOEAS{}, the validation of our proposal is carried out taking into consideration the \MOEADDE{}~\cite{zhang2009performance}, the \NSGAII{}~\cite{deb2002fast}, \REMOA{}~\cite{trautmann2013r2} and \NSGAIII{}~\cite{deb2013evolutionary} which are decomposition-based, dominance-based, indicator-based and hybrid, respectively.
%
This paper clearly shows the remarkable benefits of properly taking into account the diversity of the variable space.

The rest of this paper is organized as follows.
%
Section~\ref{Sec:LiteratureReview} provides a widely review of decomposition algorithms, diversity in EAs and related works.
%
The \VSDMOEAD{} proposal is detailed in section~\ref{Sec:Proposal}.
%
Section~\ref{Sec:ExperimentalValidation} is devoted to an extensive experimental validation of the novel proposal.
%
Finally, conclusions and some lines of future work are given in section~\ref{Sec:Conclusion}.
%
