\IEEEPARstart{M}{ulti-objective} Evolutionary Algorithms (MOEAs) are one of the most popular approaches to deal with Multi-objective Optimization Problems (MOPs).
%
MOEAs are usually employed in problems whose formulation is complicated or inaccessible.
%
A continuous box-constrained minimization MOP involves two or more conflicting objectives and are defined in Eq. (\ref{eqn:main})
	%
\begin{equation}\label{eqn:main}
\begin{split}
&min \quad F(x) = (f_1(x), ..., f_M(x)) \\
&s.t. \quad x \in \Omega.
\end{split}
\end{equation}

where $\Omega \subseteq \Re^D$ denotes the decision space, $F: \Omega \rightarrow Y \subseteq \Re^M$ consists of $M$ objectives and $Y$ is the objective space.
%
Given two solutions $x_1, x_2 \in \Re^D$ is said that $x_1$ dominates $x_2$ denoted as $x_1 \prec x_2$ if and only if $f_i(x_1) \leq f_i(x_2)$ for all $i \in \{1,...,M\}$ and $f_i(x_1) < f_i(x_2)$ for at least one objective.
%
%Therefore, $f_i(x_1)$ should be better or equal to $f_i(x_2)$ and $f_i(x_1)$ should be better for at least one objective.
%
A solution $F(x^*)$ is called a Pareto-optimal solution if there does not exist $F(x) \in Y$ such that $x \prec x^*$.
%
The set of all $x^* \in Y$ is called the Pareto-optimal solution set (PS), and their image is the Pareto Front (PF).
%
The goal of the MOEAs is to find a set of solutions that are well-distributed and converged to the PF in the objective space~\cite{trivedi2016survey}.
%

The Evolutionary Algorithms (EAs) are popular meta-heuristics to deal with MOPs due its capability to approximate several solutions in a single run.
%
In the last decade, several strategies that take into account executions in long-term have been quite successfull mainly in the most complex problems~\cite{segura2016improving}.
%
These strategies explicitly preserves the diversity in the population incorporating the stopping criterion and elapsed time to attain a properly balance between exploration and exploitation~\cite{segura2015novel}.
%

The mechanisms designed to deal with diversity have turned to be essential to attain quality solutions in single-EAs.
%
Perhaps, one of the most critical issues of promoting diversity is that it provides a way to deal with premature convergence and stagnation.
%
Diversity can be taken into account in the design of several components such as in the variation stage~\cite{herrera2003fuzzy},~\cite{mitchell1998introduction}, replacement phase~\cite{segura2015novel} and/or popultions models~\cite{koumousis2006saw}.
%

Recently, several remarkable EAs incorporates a replacement phase, which maintains a balance between exploration and exploitation.
%
Such transition is gradually imposed taking into account the stopping criterion and the elapsed time.
%
Those strategies that incorporate such replacement phase has attained remarkable results, mainly in long-term executions.
%
For instance, in combinatorial domains new best-known solutions for some well-known variants of the frequency assignment problem~\cite{segura2016improving}, and for a two-dimensional packing problem~\cite{de2010optimisation}.
%
In addition, this principle guided the design of the winning strategy at the Second Wind Fram Layout Optimization Competition, that was held in the Genetic and Evolutionary Conference.
%
Recently, in the case of continuous domains this replacement phase has been incorporated to Differential Evolution (DE)~\cite{castillo2019differential}, which attained remakably superior results than the winners of the competition carried out in IEEE Congress of Evolutionary Computation (CEC) of the years 2016 and 2017.
%
Therefore, this novel principle of incorporing the replacement phase to explicitly control the diversity has given quite good results in both discrete and continuous domains.
%
Remarkably, the incorporation of this paradigm in EAs have allowed to discover new solutions in several NP-hard problems, which required long-term executions which tend to be even more feasible caused by the constatn growing of computational power.
%

The usual design of MOEAs is especifically build to attain well-spread solutions and to cover the PF (coverage), therefore some of them incorporate different mechanisms to achieve such goal.
%
However, most of the MOEAs disregard the variable space even though those algorithms can suffer the same drawbacks raisen in single-objective space, e.g. premature convergence and stagnation.
%
Perhaps one of the main challenges to incorporate strategies to control the diversity in variable space is that in multi-objective spaces inducing diversity in variable spaces does not guarantee diversity in the objective space.
%
Following the goal of MOEAs where is desired to attain solutions well-spread in the objective space the MOEAs implicilty inherit some degree of diversity in the variable space, therefore convergence does not appear in the variable space.
%
Nevetheless, the implicit diversity induced in the variable space by the objective space might not be enough an the reproduction operators could lose its exploratiory strength.
%

In spite of the amazing amount of MOEAs that have been developed, this paper proposes a novel and different MOEA \textit{the Variable Space Diversity MOEA based in Decomposition} (VSD-MOEA/D), which explicitly induces diversity in the variable space through several stages to obtain a proper balance from exploration toward exploitation.
%
Particularly, the MOEA/D-DE that attained the first place in the CEC-09 is taken into account~\cite{zhang2009performance}.
%
This MOEA is transformed incorporating an original replacement phase that takes into consideration the stopping criterion and the number of function evaluations.
%
In this way, this algorithm grants more importance to the diversity of variable space in the initial stages, and as the function evaluations evolve, it gradually grants more importance to the diversity of the objective space.
%
Thus in the last functions evaluations the algorithm has a similar behavior than the state-of-the-art MOEAs.
%
In addition, VSD-MOEA/D employes three populations and deals with the diveristy problems caused by the mating selection and replacement mechanisms of the MOEA/D-DE.
%
Since that in the literature exists a broad kind of MOEAs based in decomposition the validation of our proposal is carried out taking into consideration the  MOEA/D~\cite{zhang2007moea} and  MOEA/D-DE~\cite{zhang2009performance} that are based in decomposition and R2-EMOA that is based in indicators.
%
This paper clearly shows the remarkable benefits of properly taking into account the diversity of the variable space.

The rest of this paper is organized as follows.
%
Section~\ref{Sec:LiteratureReview} provides a widely review of decomposition algorithms, diversity in EAs and related works.
%
The VSD-MOEA/D proposal is detailed in section~\ref{Sec:Proposal}.
%
Section~\ref{Sec:ExperimentalValidation} is devoted to the experimental validation of the novel proposal.
%
Finally, conclusions and some lines of future work are given in section~\ref{Sec:Conclusion}.
%
