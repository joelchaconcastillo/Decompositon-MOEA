\IEEEPARstart{M}{ulti-Objective} Evolutionary Algorithms (\MOEAS{}) are one of the most popular approaches 
for dealing with Multi-Objective Optimization Problems (\MOPS{})~\cite{das2011real, zhou2011multiobjective}.
%
\MOEAS{} are usually employed in complex problems where more traditional optimization techniques are not 
applicable~\cite{Lootsma:99}.
%
%TODO: tal y como lo defines no es box-constrained. Quita lo de box-constrained o mejor define las restricciones
%de caja.
A continuous box-constrained minimization \MOP{}, which is the case addressed in this paper,
involves two or more conflicting objectives as defined in (\ref{eqn:main})
%
\begin{equation}\label{eqn:main}
\begin{split}
&min \quad F(x) = (f_1(x), ..., f_M(x)) \\
&s.t. \quad x_i^{(L)} \leq x_i \leq x_i^{(U)}
%&s.t. \quad x \in \Omega.
\end{split}
\end{equation}

where $x \in \Re^D$, $D$ is the number of variables,
and each decision variable $x_i \in \Re$ is constrained by $x_i^{(L)}$ and $x_i^{(U)}$, 
i.e. the lower bound and the upper bound.
%$\Omega \subseteq \Re^D$ denotes the decision or variable space, 
The feasible space bounded by $x_i^{(L)}$ and $x_i^{(U)}$ is denoted by $\Omega$.
Each solution is mapped to the \textit{objective space} $Y$ with the function $F: \Omega \rightarrow Y \subseteq \Re^M$, 
which consists of $M$ real-valued objective functions.
%
%
%Decidir quitar estas definiciones porque la introduccion está muy larga
%
%Given two solutions $x_1, x_2 \in \Omega$, $x_1$ dominates $x_2$, denoted as $x_1 \prec x_2$, 
%if and only if $f_i(x_1) \leq f_i(x_2)\ \forall\  i \in \{1,...,M\}$ and 
%$\exists i \in \{1,...,M\}: \ f_i(x_1) < f_i(x_2)$.
%
%A solution $x^*$ is a Pareto-optimal solution if $\not \exists x \in \Omega : x \prec x^*$.
%
%The set of Pareto-optimal solutions is called the Pareto-optimal set (\PS{}), and their images are the 
%Pareto Front (\PF{}).
%
The goal of most \MOEAS{} is to find a proper approximation of the Pareto Front, i.e., a set of
solutions whose images are well-distributed and close to the Pareto Front~\cite{trivedi2016survey}.
%
%%%%%%%%%%%%%%%%%%%%%%%%%%%%%%%%%%%%%%%%%%%%%%%%%%%%% end paragraph

%Evolutionary Algorithms (\EAS{}) are one of the most popular meta-heuristics employed to tackle 
%\MOPS{}.
%
In recent years, the development of \MOEAS{} has grown dramatically~\cite{van1998multiobjective, coello2007mop}, resulting 
in effective and broadly applicable algorithms.
%
However, some function features provoke significant degradation of the performance of \MOEAS{}~\cite{huband2006review}, 
meaning better design principles are still required.
%
%TODO: Citar tras taxonomies alguna clasificacion, preferiblemente la de decomposition, dominance, quality.
Regarding the design of \MOEAS{}, several paths have been explored, resulting in diverse taxonomies~\cite{trivedi2016survey}.
%
For instance, principles related to decomposition, dominance and quality metrics are used
to design \MOEAS{}.
%
Current state-of-the-art \MOEAS{} consider in some way the diversity in the objective space.
%
In some cases, this is done explicitly through density estimators~\cite{beume:07}, %TODO: Citar algun density estimator
whereas in other cases,
%such as in decomposition-based methods 
this is done indirectly~\cite{zhang2007moea}. 
%with the use of a diverse set of weights. %TODO: Citar MOEA/D
%
Since optimizing most objective space quality indicators implies attaining a well-spread set of solutions in the
objective space, not considering this kind of diversity would result in fairly ineffective optimizers.
%
A quite different condition appears with respect to diversity in the variable space.
%
Since objective space quality metrics do not consider at all the diversity in the variable space,
most \MOEA{} designers disregard this diversity.

Alternatively, %when looking at the best optimizers for other areas of optimization, such as i
%the single-objective case,
%it is clear that 
several state-of-the-art single-objective methods introduce mechanisms to vary the trend of the 
diversity in the variable space, even if obtaining a diverse
set of solutions is not the aim of the optimization~\cite{Joel:Crepinsek}.
%
Instead, this is done to induce a better balance between exploration and exploitation.
%
In fact, the proper management of diversity is considered one of the cornerstones for proper performance~\cite{Herrera-Poyatos:17}.
%
Thus, these differences between the design principles applied in single-objective and multi-objective evolutionary 
algorithms are surprising.
%
Moreover, practitioners have shown that modern \MOEAS{} suffer some drawbacks involving stagnation and premature 
convergence in subsets of variables~\cite{ishibuchi2006comparison, castillo2017multi, buche2003self, lu2002dynamic}.
%
As a result, this paper studies the hypothesis that incorporating mechanisms to manage the diversity in the variable space 
might yield important benefits to the field of multi-objective optimization.
%
%TODO: citar ejemplos significativos de niching-based MOEAS{}.
Note that unlike other proposals, such as niching-based \MOEAS{}~\cite{mahfoud1995niching, srinivas1994muiltiobjective}, we are not interested in obtaining
a diverse set of solutions in the variable space; rather, we state that the quality of the results in terms of objective-space indicators can be improved further with these
kinds of mechanisms.

Since controlling diversity in the variable space is 
so important in single-objective domains to attain a proper balance between exploration and intensification~\cite{lin2009auto},
a large number of related methods have been devised~\cite{Joel:Crepinsek}. 
%
%Esto esta interesante, pero muy largo, lo pasare a la parte de literatura
%Note that in most cases, similar principles have been used to devise niching methods that try to locate
%several optima simultaneously and methods that try to properly balance exploration and intesification but with the aim
%of attaining a single optimum.
%
%However, since the aim is quite different, methods that are specific to one of these areas
%have also been proposed.
%
%The applied principles to vary the way diversity evolves are quite assorted so several taxonomies have been proposed.
%
%They are classified in~\cite{Joel:Crepinsek} in terms of the involved component that is altered.
%
%Some of the most prominent categories that are identified are the following: 
%selection-based~\cite{Chen:09}, 
%population-based\cite{koumousis2006saw}, 
%crossover/mutation-based~\cite{herrera2003fuzzy}, %mitchell1998introduction
%fitness-based~\cite{Cioppa:07}, 
%and replacement-based~\cite{segura2015novel}.
%
Recent research on single-objective optimization has shown that important advances can be achieved when the
balance between exploration and intensification is managed by relating the diversity of the population to 
the stopping criterion and the elapsed execution time.
%
Specifically, these methods reduce the importance given to preserving diversity as the end of the optimization
is approached.
%
This principle has been used to find new best-known solutions for the Frequency Assignment Problem 
(FAP)~\cite{segura2016improving}, to improve further continuous optimizers~\cite{castillo2019differential} and
to design the winning strategy in the extended round of Google Hash Code 
2020\footnote{https://codingcompetitions.withgoogle.com/hashcode/}, which featured over $100,000$ participants.
%
Thus, we decided to explore the incorporation of this principle into the design of \MOEAS{}.
%
%Probably, one of the most important drawbacks of this design principle is that most discussed optimizers require long-term
%executions to excel.
%
%Even if this is the case for \MOEAS{}, there is a niche for these kinds of optimizers, so exploring them is important.
%Toda la parte de detalles de Evolución Diferencial se me hacían demasiado largas, tal vez para la revisión de literatura sí
%está bien.
%
%%%%%%%%%%%%%%%%%%%%%%%%%%%%%%%%%%%%%%%%%%%%%%%%%%%%% end paragraph

One of the main problems in incorporating the above principle into the design of \MOEAS{} is that
measures of the variable and objective spaces must be considered simultaneously.
%and the relation between them is problem-dependent.
%
This design principle is based on reducing the importance of diversity in the variable space as 
generations evolve, so we maintain this decision and indirectly increase the importance granted to diversity 
and quality in the objective space as the execution progresses.
%
%Esta es demasiado detalle asi que mejor para la propuesta
%In fact, our proposal behaves as a traditional \MOEA{} during the last half of the optimization, in the sense
%that the diversity on the variable space is disregarded.
%
In order to show the validity of our hypothesis, this paper proposes the
\textit{Archived Variable Space Diversity MOEA based on Decomposition} (\AVSDMOEAD{}).
%, which explicitly manage the diversity in the 
%variable space to properly balance from exploration towards intensification.
%
\AVSDMOEAD{} simplifies \MOEADDE{}~\cite{zhang2009performance} by deactivating the dynamical resource allocation
scheme and disregarding the notion of neighborhood;
 at the same time, it is extended by including a novel replacement strategy that applies the design principles discussed.
%
Our proposal is validated by taking into consideration 
\MOEADDE{}~\cite{zhang2009performance}, 
\NSGAII{}~\cite{deb2002fast}, 
\REMOA{}~\cite{trautmann2013r2} and 
\NSGAIII{}~\cite{deb2013evolutionary}.
%Esto lo pasamos para la validación con el fin de acortar
%which are popular representatives of decomposition-based, dominance-based, indicator-based and hybrid \MOEAS{}, respectively.
%
Remarkable benefits are achieved in terms of robustness and scalability.

The rest of this paper is organized as follows.
%
Section~\ref{Sec:LiteratureReview} provides a review of \MOEAS{}, diversity management 
and other related works.
%
The \AVSDMOEAD{} proposal is detailed in section~\ref{Sec:Proposal}.
%
Section~\ref{Sec:ExperimentalValidation} is devoted to an extensive experimental validation of the novel proposal and
design principle.
%
Finally, conclusions and some lines of future work are presented in section~\ref{Sec:Conclusion}.
%
