In this section a decomposition \MOEA{} that promotes diversity in the decision variable space is proposed.
%
Initially the main framework of the propose \MOEA{}, which follows the decomposition \MOEAS{} style is described.
%
Thereafter, the incorpored replacement phase to preserve diversity in decision variable space is explained in detail.
%
Finally, several considerations --probed empirically-- related to the \DE{} operator are disscused.
%


The idea behind our proposal is to boost a proper balance between exploration and exploitation.
%
In particular the \textit{Variable Space Diversity \MOEA{} based in Decomposition} (\VSDMOEAD{}) explores properly the search space decreasing gradually the diversity requirements of the population in function of the elapsed generations and stopping criterion.
%
In this way, \VSDMOEAD{} explicitly grants diversity in the variable space and preservers quality solutions in the objectives space.
%

The two novelties of \VSDMOEAD{} are the inclusion of a replacement phase and Elite population.
%
While the former preserves diversity keeping the neighbourhoods definitions, the latter records the best individual in each subproblem.
%
Particularly, the Parent population ($P$), Offspring population ($Q$) and the Elite population ($E$) are taken into consideration.
%
%
The role of the Parent population ($P$) is to provide diverse solutions at several stages in function of the elapsed time.
%
In contrast, the Elite population ($E$) records the best values attained in each subproblem among all the optimization process.
%


The general framework of the \VSDMOEAD{} is taken from \MOEADDE{}~\cite{li2009multiobjective} which employes \DE{} operators.
%
Algorithm \ref{alg:vsd-moea} shows the main procedure of \VSDMOEAD{}.
%
Similarly than \MOEADDE{} the initialization create a random population, assign an adequately ideal vector $\mathbf{z}$, and the neighborhoods of each subproblem are associated (line \ref{alg_1:1}).
%
Then iterativelly the generations evolve until the stopping criterion is met (lines \ref{alg_1:3} - \ref{alg_1:9}).
%
Particularly, in each generation and for each subproblem the following steps take place.
%
In the mating selection three different indexes ($r_1, r_2, r_3$) are randomly selected, those indexes belong to the neighborhood $B(i)$ with probability $\delta$ or to the entire population with probability $(1 - \delta)$ (line \ref{alg_1:4}).
%
In the variation stage (line \ref{alg_1:5}) a new solution $Q^t_i$ is created employing the individuals $P^t_{r_1}, P^t_{r_2}, P^t_{r_3}$ with the application of \DE{} and polynomial mutation CITA.
%
In line \ref{alg_1:6} the reference vector $\mathbf{z}$ is updated as the lowest objective values. 
%
Given the new individual $Q^t_i$ all the elite individuals belonging to the neighborhood $B(i)$ are updated according to the a specific approach i.e. Tchebycheff approach.
%
Finally, the Parents ($P^t$), Elite ($E_t$) and Offspring ($Q^t$) populations are joined to select the parents of the next generation ($P^{t+1}$)  in the replacement (line \ref{alg_1:7}).
%
Although that the three populations are joinned and treated in the replacement stage (detailed in the following section) the order of the individuals is important, therefore the parents $P^{t+1}$ of next generations preserves the order that is related with each subproblem.
%




\subsection{Replacement Phase of \VSDMOEAD{} }


\subsection{ Additional considerations}


*Describe some about the mating and the replacement and get in detail about the diversity drawbacks with the proposed strategies of the state-of-the-art algortihms.
*Introduce in some way the three populations.
*Diagram support.
*Explain in detail the pseudocode.
*Subsection for the replacment phase.
*

\begin{algorithm}[!t]
\algsetup{linenosize=\tiny}
        \caption{Main procedure of VSD-MOEA/D} 
        \begin{small}
\begin{algorithmic}[1]
	\STATE \textbf{Initialization}: Generate an initial population $P^0$ with $N$ individuals, initialize $\mathbf{z} = (z_1, ..., z_m)^T$ far away from the front, initialize the weight vectors $\lambda^1, ..., \lambda^N $ and neighbourhoods $B(i)$. \label{alg_1:1}
%	\STATE \textbf{Evaluation}: Evaluate all individuals of $P^0$ according to the Tchebycheff approach.
        \STATE Assign $t=0$ \label{alg_1:2}
        \WHILE{ (not stopping criterion)  } \label{alg_1:3}
	   \FOR{ each subproblem $i \in N$}
               \STATE \textbf{Mating selection}: Select randomly three indexes ($r_1 \neq r_2 \neq r_3 \neq i$) from neighborhood $B(i)$ with probability $\delta$ or from the entire population with probability $(1-\delta)$. \label{alg_1:4}
%%	       Fill the mating pool by performing binary tournament selection on $P_t$, 
  %                   based on the non-dominated ranks (ties are broken randomly).
	       \STATE \textbf{Variation}: Generate a solution $y$ from $P^t_{r_1}$, $P^t_{r_2}$ and $P^t_{r_3}$ by \DE{} operator, and perform a mutation operator on $y$ with probability $p_m$ to produce a new solution $Q^t_{i}$. \label{alg_1:5}
	       %Apply SBX crossover and Polynomial mutation to the mating pool to create a child population $Q_t$.
	 %      \STATE \textbf{Evaluation}:  Evaluate individual $Q^t_i$.
%Evaluate $F(Q^t_{i})$ and $g(x | \lambda, z)$according to the Tchebycheff approach.
	       \STATE \textbf{Update reference} $\mathbf{z}$: For each $j=1, .., m$, set $z_j = min\{z_j, F_j(Q^t_i) \}$ \label{alg_1:6}
	       \STATE \textbf{Survivor selection}: Update all the elite vectors ($E^t$) from the neighborhood $B(i)$ according $g(E^t_i | \lambda_i, z)$. \label{alg_1:6}
	   \ENDFOR
	   \STATE \textbf{Replacement}: Generate $P^{t+1}$ by applying the replacement scheme described in  Algorithm \ref{alg:replacement}, using $P^t$, $Q^t$ and $E^t$ as inputs. \label{alg_1:7}
           \STATE $t=t+1$ \label{alg_1:8}
        \ENDWHILE \label{alg_1:9}
        \end{algorithmic}
        \end{small}
\label{alg:vsd-moea}
\end{algorithm}




\begin{algorithm}[t]
\algsetup{linenosize=\tiny}
        \caption{Replacement Phase of VSD-MOEA/D}
\begin{small}
\begin{algorithmic}[1]
\STATE Input: $P^t$ (Parent population), $Q^t$ (Offspring population), $E^t$ (Elite population).
        \STATE Output: $P^{t+1}$
        \STATE $R^t = P^t \cup Q^t \cup E^t$ (Keeping their associated subproblems) \label{alg:1} 
        \STATE $P^{t+1} = \emptyset$ \label{alg:2}
        \STATE $Penalized = \emptyset$ \label{alg:3}
        \STATE $D^t = D_I - D_I * \frac{G_{Elapsed}}{0.5*G_{End}}$ \label{alg:4}
        \WHILE{ A subproblem is not assigned } \label{alg:6}
            \STATE Compute $DCS$ of individuals in $R^t$, using $P^{t+1}$ as a reference set \label{alg:7}
            \STATE Move the individuals in $R^t$ with $DCS < D^t$ to $Penalized$ \label{alg:8}
                \IF{$R^t$ is empty} \label{alg:9}
                    \STATE Compute $DCS$ of individuals in $Penalized$, using $P^{t+1}$ as a reference set \label{alg:10}
                    \STATE Move the individual in $Penalized$ with the largest $DCS$ to $R^t$ \label{alg:11}
                \ENDIF
		\STATE Select a new survivor from $R^t$ with the best function value (associated with its weigth vector) and move it to its related subproblem in $P^{t+1}$.
		\STATE 
	
        \ENDWHILE
        \RETURN $P^{t+1}$ \label{alg:14}
        \end{algorithmic}
\end{small}
\label{alg:replacement}
\end{algorithm}


\begin{equation}\label{eqn:distance}
Distance(A, B) =   \left ( \frac{1}{n}  \sum_{i=1}^n \left ( \frac{A_i - B_i}{x_i^{(U)} - x_i^{(L)}} \right )^2  \right)^{1/2}
\end{equation}


