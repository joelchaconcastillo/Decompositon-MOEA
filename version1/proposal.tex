This section is devoted to describe a novel \textit{Variable Space Diversity \MOEA{} based in decomposition} (\VSDMOEAD{}) that improves further more the results attained by the-state-of-the-art \MOEAS{}.
%
\VSDMOEAD{} could give quite competittive results taking into account middle-term executions, however its best performance is attained in long-term executions.
%
Additionally, under minor changes, this \MOEA{} can be employed to discrete domains.
%
One of the goals in the development of our proposal is to provide a way to attain quality solutions in \MOPS{} incorporating the elapsed time and the stopping criterion.


%
In inception, \MOEAD{}~\cite{zhang2007moea} incorporated an elitism design which provoked several diversity short-comings.
%
Given this, several improvements were stablished to overcome this drawback~\cite{trivedi2016survey}.
%
For instance, \MOEADDE{}~\cite{li2009multiobjective} incorporated several measures in the mating selection and replacement mechanism.
%
Nevertheless, those strategies suffered of overparametrization, being quite sensible to each problem.
%
In contrast, \VSDMOEAD{} employes a mechanism to explicitly promote diversity to attain a balance from exploration towards exploitation among the execution.
%
This transition is gradually accomplished exploring adequately the decision variable space.
%
Although several \MOEAS{} have been proposed to promote diversity in the decision variable space (Section \ref{MOEAs:Diversity}), none of them have attained better results than the-state-of-the-art-\MOEAS{}.
%
Perhaps the main reason is that those \MOEAS{} do not employed a linear dynamical model to induce a balance between exploration towards exploitation.
%


The novelties of \VSDMOEAD{} is the incorporation of a sophisticated replacement phase and the Elite population, which participates in the mating selection.
%
Particularly, the Parent population $P$, Child population $Q$ and the Elite population $E$ are taken into consideration.
%
Each population is ordered by subproblem.
%
The roll of the Parent population ($P$) is to maintain and generate diverse solutions ($Q$) to properly explore the search space to several stages.
%
In contrast the Elite population ($E$) records the best values attained by each subproblem among all the optimization process.
%

In the literature several enhancements to the basic \MOEAD{} are proposed.
%
Mainly, a line of research related to the mating selection and the replacement mechanism was taken.
%
Particularly, this line is highly related to the diversity maintined in the population.
%
However, since that \VSDMOEAD{} incorporates a new replacement mechanism and based to several experiments the mating selection and replacement mechanism of the classic \MOEAD{} have been removed.





*Describe some about the mating and the replacement and get in detail about the diversity drawbacks with the proposed strategies of the state-of-the-art algortihms.
*Introduce in some way the three populations.
*Diagram support.
*Explain in detail the pseudocode.
*Subsection for the replacment phase.
*

\begin{algorithm}[!t]
\algsetup{linenosize=\tiny}
        \caption{Main procedure of VSD-MOEA/D} 
        \begin{small}
\begin{algorithmic}[1]
	\STATE \textbf{Initialization}: Generate an initial population $P_0$ with $N$ individuals, initialize $z = (z_1, ..., z_m)^T$ far away from the front.
	\STATE \textbf{Evaluation}: Evaluate all individuals of $P$ according the decomposition method.
        \STATE Assign $t=0$
        \WHILE{ (not stopping criterion)  }
	   \FOR{i = 1,...,N}
               \STATE \textbf{Mating selection}: Set $r_1 = i$ and randomly select two different indexes $r_2$ and $r_3$ from $P$.	       
%%	       Fill the mating pool by performing binary tournament selection on $P_t$, 
  %                   based on the non-dominated ranks (ties are broken randomly).
	       \STATE \textbf{Variation}: Generate a solution $y$ from $P_{r_1}$, $P_{r_2}$ and $P_{r_3}$ by \DE{} operator, and perform a mutation operator on $y$ with probability $p_m$ to produce a new solution $Q_{r_1}$.
	       %Apply SBX crossover and Polynomial mutation to the mating pool to create a child population $Q_t$.
	       \STATE \textbf{Evaluation}: Evaluate $F(Q_{r_1})$.

	       \STATE \textbf{Survivor selection}: Update the elite vectors
               \STATE $t=t+1$
	   \ENDFOR
	   \STATE \textbf{Replacement}: Select the $P$ vectors according...
        \ENDWHILE
        \end{algorithmic}
        \end{small}
\label{alg:vsd-moea}
\end{algorithm}




\begin{algorithm}[t]
\algsetup{linenosize=\tiny}
        \caption{Replacement Phase of VSD-MOEA}
\begin{small}
\begin{algorithmic}[1]
\STATE Input: $P_t$ (Population of current generation), $Q_t$ (Offspring of current Generation)
        \STATE Output: $P_{t+1}$
        \STATE $R_t = P_t \cup Q_t$ \label{alg:1}
        \STATE $P_{t+1} = \emptyset$ \label{alg:2}
        \STATE $Penalized = \emptyset$ \label{alg:3}
                                \STATE $D_t = D_I - D_I * \frac{G_{Elapsed}}{0.5*G_{End}}$ \label{alg:4}
        \WHILE{ $|P_{t+1}|$ $\leq$ N } \label{alg:6}
                                        \STATE Compute $DCS$ of individuals in $R_t$, using $P_{t+1}$ as a reference set \label{alg:7}
                                        \STATE Move the individuals in $R_t$ with $DCS < D_t$ to $Penalized$ \label{alg:8}
                \IF{$R_t$ is empty} \label{alg:9}
                                                \STATE Compute $DCS$ of individuals in $Penalized$, using $P_{t+1}$ as a reference set \label{alg:10}
                                                \STATE Move the individual in $Penalized$ with the largest $DCS$ to $R_t$ \label{alg:11}
                \ENDIF
                                        \STATE Identify the first front ($F$) in $R_t \cup P_{t+1}$ with an individual $I \in R_t$ \label{alg:12}
                                        \STATE Use the novel density estimator (Algorithm~\ref{alg:Density_Estimator}) to select a new survivor
                                        from $F$ and move it to $P_{t+1}$\label{alg:13}
        \ENDWHILE
        \RETURN $P_{t+1}$ \label{alg:14}
        \end{algorithmic}
\end{small}
\label{alg:Replacement_Phase}
\end{algorithm}


\begin{equation}\label{eqn:distance}
Distance(A, B) =   \left ( \frac{1}{n}  \sum_{i=1}^n \left ( \frac{A_i - B_i}{x_i^{(U)} - x_i^{(L)}} \right )^2  \right)^{1/2}
\end{equation}


