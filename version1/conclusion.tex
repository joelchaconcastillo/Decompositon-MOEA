Premature convergence is one of the most typical drawbacks of \EAS{}.
%
In many \MOPS{}, \MOEAS{} promote indirectly the preservation of diversity on variable space
because of the explicit maintenance of diversity on objective space.
%
However, for many problems the degree of maintained diversity is not enough to keep the exploration
power and locate the optimal regions.
%
In single-objective optimization, many of the state-of-the-art algorithms consider an explicit control of
diversity.
%
Particularly, those schemes to relate the degree of diversity with the elapsed period and stopping criterion
have excelled.
%
This paper shows that this design principle is also important for the multi-objective optimization area,
where the optimization of many of the most complex popular benchmark problems can be improved further
by introducing this design principle.

In order to prove previous hipothesis, a novel replacement operator based on the aforementioned design principle
is integrated to generate a decomposition-based \MOEA{} that takes into account the diversity of both variable 
space and objective function space.
%
This is performed using a dynamic penalty method. 
%
Note that since the aim of the approach is to improve the results when considering metrics of the objective space,
the importance given to the diversity on the variable space is reduced as the evolution progress, meaning that
al later phases our proposal behaves more similarly to traditional \MOEAS{}.
%
Additionally, taking into account some of the recent advances in the area, an external archive based on the
R2-indicator is incorporated.
%
As a result our proposal is referred to as \AVSDMOEAD{}.
%

The experimental validation carried out shows the remarkable improvement provided by \AVSDMOEAD{} in comparison to
state-of-the-art \MOEAS{} both in two-objective and three-objective problems.
%
The scalability analyses shows that as the number of decision variables increases the benefits of including
a proper management of diversity is even more important, so the differences in performance increases.
%
Additionally, the analysis of the initial threshold distance, which is an additional parameter required by \AVSDMOEAD{}, 
shows that the method is quite robust so finding a proper parameter value is not a difficult task.
%
Finally, in order to better understand the reasons behind the huge superiority of the proposal, some analyses regarding
the dynamics of the populations are provided.

In the future we plan to apply the principles studied in this paper to other categories of \MOEAS{}.
%
For instance, including the diversity management put forth in this paper in decomposition-based and indicator-based \MOEAS{} seems plausible.
%
Additionally, we would like to develop an adaptive scheme to avoid setting the initial threshold value.
%
Finally, in order to obtain even better results, these strategies are going to be incorporated with continuation and/or individual
improvement methods.
