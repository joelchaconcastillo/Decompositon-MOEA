Premature convergence is one of the most typical drawbacks of \EAS{}.
%
\MOEAS{} promote indirectly the preservation of diversity on variable space
because of the explicit maintenance of diversity on objective space.
%
However, for many problems the degree of diversity maintained is not enough to keep the exploration
power and locate the optimal regions.
%
In single-objective optimization, many of the state-of-the-art algorithms consider an explicit management of
diversity.
%
Particularly, those schemes that relate the degree of diversity with the elapsed period and stopping criterion
have excelled.
%
This paper shows that this design principle is also helpful for the multi-objective optimization area,
where the optimization of many of the most complex popular benchmark problems can be improved further
by aplying this design principle.

In order to prove previous hypothesis, a novel replacement operator based on the aforementioned design principle
is integrated to generate a decomposition-based \MOEA{} that takes into account the diversity of both variable 
space and objective function space.
%
This is performed using a dynamic penalty method. 
%
Note that since the aim of the approach is to improve the results when considering metrics of the objective space,
the importance given to the diversity on the variable space is reduced as the evolution progresses, meaning that
at later phases our proposal behaves more similarly to traditional \MOEAS{}.
%
Additionally, taking into account some of the recent advances in the area and that our proposal might not maintain
some high-quality solutions in the population because of the penalty scheme, an external archive based on the
R2-indicator is incorporated.
%
As a result our proposal is referred to as \AVSDMOEAD{}.
%

The experimental validation carried out shows the remarkable improvement provided by \AVSDMOEAD{} in comparison to
state-of-the-art \MOEAS{} both in two-objective and three-objective problems.
%
The scalability analyses show that as the number of decision variables increases the benefits of including
a proper management of diversity is even more important, so the differences in performance increases.
%
In fact, the most remarkable benefits appear for the most complex cases.
%
Additionally, the analysis of the initial penalty threshold, which is an additional parameter required by \AVSDMOEAD{}, 
shows that the method is quite robust so finding a proper parameter value is not a difficult task.
%
Finally, in order to better understand the reasons behind the huge superiority of the proposal, some analyses regarding
the dynamics of the populations are provided.

In the future we plan to apply the principles studied in this paper to other categories of \MOEAS{}.
%
For instance, including the diversity management put forth in this paper in indicator-based \MOEAS{} seems plausible.
%
Additionally, in order to obtain even better results, these strategies are going to be incorporated with continuation and/or individual
improvement methods.
