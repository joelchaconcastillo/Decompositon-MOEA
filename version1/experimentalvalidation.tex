This section describes the experimental validation carried out to study the performance and
gain a clear understanding of the specifics of \VSDMOEA{}.
%
Our results clearly show that controlling the diversity of the variable space provides a way to further improve the results
obtained by the state-of-art \MOEAS{}.
%
First, we discuss some technical specifications involving the benchmark problems and algorithms implemented.
%
We then present a comparison between \VSDMOEA{} and state-of-the-art algorithms when used on the long-term.
%
Then, three additional experiments to fully validate \VSDMOEA{} are included.
%
These analyses are designed to test the scalability in the variable space, the performance with different stopping criteria,
and the behavior with different initial penalty thresholds.

This work takes into account some of the most popular and widely used benchmarks in the multi-objective field.
%
These problems are the WFG~\cite{Joel:WFG}, DTLZ~\cite{Joel:DTLZ}, and UF~\cite{Joel:CEC2009} configured in a
standard way.
%
The WFG test problems were used with two and three objectives and
were configured with $24$ parameters, $20$ of them corresponding to distance parameters and $4$ to position parameters.
%
In the DTLZ test problems, the number of variables was set to $n=M+r-1$, where $r=\{5, 10, 20\}$ for DTLZ1, DTLZ2 to DTLZ6 and DTLZ7, respectively.
% 
The UF benchmark comprises seven problems with two objectives (UF1-7) and three problems with three objectives (UF8-10).
%
All of them were configured with $30$ variables.
%
Note that the experiment used to analyze the scalability in the variables considers different numbers of variables.

The experimental validation includes three well-known state-of-the-art \MOEAS{} and \VSDMOEA{}.
%
The \MOEAS{} that are considered are \NSGAII{}~\cite{Joel:jMetal}, \MOEAD{}~\cite{MOEADCode}, and \RMOEA{}~\cite{R2EMOACode},
which can be classified as dominance-based, decomposition-based, and indicator-based, respectively.
%
In the case of \MOEAD{}, several variants have been devised.
%
The \MOEAD{} implementation considered is the one that obtained first place in the Congress on Evolutionary Computation's
2009 MOP Competition~\cite{zhang2009performance}.
%
The common configuration in all the experiments was as follows: the population size was set to $100$, and the genetic operators were the Simulated Binary Crossover (SBX) and polynomial
mutation~\cite{Joel:SBX1994, Joel:Mutation}.
%
The crossover probability was set to $0.9$ and the crossover distribution index was set to $2$.
%
Similarly, the mutation probability and distribution index were fixed to $1/n$ and $50$, respectively.
%
The additional parameterization required by each algorithm is shown in Table~\ref{tab:Parametrization}.
%
Note that scalarization functions are required in \MOEAD{} and \RMOEA{}.
%
In both cases, the Tchebycheff approach is used.
%
The procedure for generating the weight vectors differs in \MOEAD{} and \RMOEA{}.
%
\RMOEA{} was applied with $501$ and $496$ weight vectors for two and three objectives, respectively~\cite{trautmann2013r2}.
%
In contrast, \MOEAD{} requires the same number of weight vectors as the population size.
%
They were generated with the uniform design (UD) and the good lattice point (GLP) method~\cite{Joel:MOEAD_Uniform_Design, Joel:Kuhn_Munkres}.

Given that all the algorithms considered are stochastic, each execution was repeated $35$ times with different seeds.
%
The hypervolume indicator (\HV{}) is used to compare the various schemes.
%
Note that in the supplementary material, the results are also compared in terms of the IGD+ metric, with the conclusions being quite similar.
%
The reference point used to calculate the \HV{} is chosen to be a vector whose values are sightly larger (ten percent) than the nadir point,
as suggested in~\cite{ishibuchi2017reference}.
%
The normalized \HV{} is used to facilitate the interpretation of the results~\cite{li2015evolutionary},
and the value reported is computed as the ratio between the normalized \HV{} obtained and the maximum attainable
normalized \HV{}.
%
In this way, a value equal to one means a perfect approximation.
%
Note that a value equal to one is not attainable because \MOEAS{} yields a discrete approximation.
%
Finally, in order to statistically compare the \HV{} ratios, a guideline similar to that proposed in~\cite{Joel:StatisticalTest} was used.
%
First a Shapiro-Wilk test was performed to check if the values of the results followed a Gaussian distribution.
%
If so, the Levene test was used to check for the homogeneity of the variances.
%
If the samples had equal variance, an ANOVA test was done; if not, a Welch test was performed.
%
For non-Gaussian distributions, the non-parametric Kruskal-Wallis test was used to test whether samples are drawn from the same distribution.
%
An algorithm $X$ is said to beat algorithm $Y$ when the differences between them are statistically significant, and the mean and median \HV{} ratios
obtained by $X$ are higher than the mean and median achieved by $Y$.



\subsection{Comparison Against State-of-the-art}
\subsection{Effect of the Initial Distance Factor}
